\documentclass[letterpaper,11pt]{article}

\usepackage{latexsym}
\usepackage[empty]{fullpage}
\usepackage{titlesec}
\usepackage{marvosym}
\usepackage[usenames,dvipsnames]{color}
\usepackage{verbatim}
\usepackage{enumitem}
\usepackage[hidelinks]{hyperref}
\usepackage{fancyhdr}
\usepackage[english]{babel}
\usepackage{tabularx}
\usepackage{fontawesome5}
\usepackage{multicol}
\setlength{\multicolsep}{-3.0pt}
\setlength{\columnsep}{-1pt}
\input{glyphtounicode}


%----------FONT OPTIONS----------
% sans-serif
% \usepackage[sfdefault]{FiraSans}
% \usepackage[sfdefault]{roboto}
% \usepackage[sfdefault]{noto-sans}
% \usepackage[default]{sourcesanspro}

% serif
% \usepackage{CormorantGaramond}
% \usepackage{charter}


\pagestyle{fancy}
\fancyhf{} % clear all header and footer fields
\fancyfoot{}
\renewcommand{\headrulewidth}{0pt}
\renewcommand{\footrulewidth}{0pt}

% Adjust margins
\addtolength{\oddsidemargin}{-0.6in}
\addtolength{\evensidemargin}{-0.5in}
\addtolength{\textwidth}{1.19in}
\addtolength{\topmargin}{-.7in}
\addtolength{\textheight}{1.4in}

\urlstyle{same}

\raggedbottom
\raggedright
\setlength{\tabcolsep}{0in}

% Sections formatting
\titleformat{\section}{
  \vspace{-4pt}\scshape\raggedright\large\bfseries
}{}{0em}{}[\color{black}\titlerule \vspace{-5pt}]

% Ensure that generate pdf is machine readable/ATS parsable
\pdfgentounicode=1

%-------------------------
% Custom commands
\newcommand{\resumeItem}[1]{
  \item\small{
    {#1 \vspace{-2pt}}
  }
}

\newcommand{\classesList}[4]{
    \item\small{
        {#1 #2 #3 #4 \vspace{-2pt}}
  }
}

\newcommand{\resumeSubheading}[4]{
  \vspace{-2pt}\item
    \begin{tabular*}{1.0\textwidth}[t]{l@{\extracolsep{\fill}}r}
      \textbf{#1} & \textbf{\small #2} \\
      \textit{\small#3} & \textit{\small #4} \\
    \end{tabular*}\vspace{-7pt}
}

\newcommand{\resumeSubSubheading}[2]{
    \item
    \begin{tabular*}{0.97\textwidth}{l@{\extracolsep{\fill}}r}
      \textit{\small#1} & \textit{\small #2} \\
    \end{tabular*}\vspace{-7pt}
}

\newcommand{\resumeProjectHeading}[2]{
    \item
    \begin{tabular*}{1.001\textwidth}{l@{\extracolsep{\fill}}r}
      \small#1 & \textbf{\small #2}\\
    \end{tabular*}\vspace{-7pt}
}

\newcommand{\resumeSubItem}[1]{\resumeItem{#1}\vspace{-4pt}}

\renewcommand\labelitemi{$\vcenter{\hbox{\tiny$\bullet$}}$}
\renewcommand\labelitemii{$\vcenter{\hbox{\tiny$\bullet$}}$}

\newcommand{\resumeSubHeadingListStart}{\begin{itemize}[leftmargin=0.0in, label={}]}
\newcommand{\resumeSubHeadingListEnd}{\end{itemize}}
\newcommand{\resumeItemListStart}{\begin{itemize}}
\newcommand{\resumeItemListEnd}{\end{itemize}\vspace{-5pt}}

%-------------------------------------------
%%%%%%  RESUME STARTS HERE  %%%%%%%%%%%%%%%%%%%%%%%%%%%%


\begin{document}

%----------HEADING----------
% \begin{tabular*}{\textwidth}{l@{\extracolsep{\fill}}r}
%   \textbf{\href{http://sourabhbajaj.com/}{\Large Sourabh Bajaj}} & Email : \href{mailto:sourabh@sourabhbajaj.com}{sourabh@sourabhbajaj.com}\\
%   \href{http://sourabhbajaj.com/}{http://www.sourabhbajaj.com} & Mobile : +1-123-456-7890 \\
% \end{tabular*}

\begin{center}
    {\Huge \scshape Kauan Izidoro} \\ \vspace{1pt}
    Sorocaba, Brasil \\ \vspace{1pt}
    \small \raisebox{-0.1\height}\faPhone\ (15) 99618-9346 ~ \href{mailto:x@gmail.com}{\raisebox{-0.2\height}\faEnvelope\  \underline{cnttkauan@gmail.com}} ~ 
    \href{https://linkedin.com/in//}{\raisebox{-0.2\height}\faLinkedin\ \underline{linkedin.com/in/kauanizidoro}}  ~
    \href{https://github.com/}{\raisebox{-0.2\height}\faGithub\ \underline{github.com/KauanIzidoro}}
    \vspace{-8pt}
\end{center}


%-----------EDUCATION-----------
\section{Educação}
  \resumeSubHeadingListStart
    \resumeSubheading
      {Senai São Paulo}{Jan. 2024 - Dez. 2025}
      {Superior em Análise e Desenvolvimento de Sistemas}{Sorocaba, São Paulo}
      \resumeSubheading
      {Senai São Paulo}{Jul. 2022 - Jul. 2024}
      {Técnico em Eletroeletrônica}{Sorocaba, São Paulo}
  \resumeSubHeadingListEnd

%------ABOUT-------
\section{Sobre}
    \resumeSubHeadingListStart
        \small{Desenvolvedor Backend com foco na criação e otimização de aplicações que solucionam desafios reais, utilizando tecnologias modernas e boas práticas de engenharia de software. Possui experiência no desenvolvimento de soluções baseadas em microsserviços, integração de sistemas Web, Desktop, IoT e implementação de arquiteturas distribuídas, com domínio em tecnologias como Python (FastAPI, Django e Flask), TypeScript (NestJS e NodeJS), \texttt{C\#} (ASP.NET), Java (Spring), Google Cloud Platform (GCP), Amazon Web Service (AWS) e tecnologias voltadas ao desenvolvimento IIoT (C, C++, Raspberry, ESP-32, MQTT, OPC-UA e OCPP).

        Expertise em bancos de dados relacionais (MySQL, PostgreSQL e SQL Server) e NoSQL (MongoDB, Apache Cassandra), além de experiência em práticas de DevOps, Cloud Computing, Linux, Docker, Kubernetes e integração de sistemas via APIs (REST, GraphQL e WebSocket). Atuação voltada para a entrega de soluções escaláveis, eficientes e alinhadas às regras de negócios.
        }
   \resumeSubHeadingListEnd

%-----------EXPERIENCE-----------
\section{Experiência}
  \resumeSubHeadingListStart

    \resumeSubheading
      {2RP Net Data Driven Company}{Jan. 2024 - Presente}
      {Desenvolvedor Back-End}{Brasil (Remoto)}
      \resumeItemListStart
        \resumeItem{Atuo na implementação de microsserviços utilizando Python e FastAPI para facilitar a comunicação entre serviços internos e APIs como BigQuery, Vertex AI, Pub/Sub e Secret Manager na Google Cloud Platform (GCP).}
        \resumeItem{Integro a equipe de migração de um ambiente em nuvem, criando um ambiente Sandbox para testes de rotinas analíticas, visando agilizar a validação de novas abordagens e a entrega de soluções aos clientes.}
        \resumeItem{Uso tecnologias de LLM/IA, como a Gemini API e Gemini SDK, para desenvolver Agentes de IA, visando automatizar a documentação/tarefas triviais e melhorar a acessibilidade às informações de projetos internos e  ganhar agilidade durante sprints.}
        \resumeItem{Utilizo práticas de DevOps, como Docker e Kubernetes, para garantir a escalabilidade, disponibilidade e eficiência das aplicações desenvolvidas.}
        \resumeItem{Participo da integração de sistemas utilizando APIs REST e WebSocket, garantindo uma comunicação ágil e eficiente entre serviços.}
      \resumeItemListEnd

    \resumeSubheading
      {Robert Bosch VM}{Jan. 2022 - Jan. 2024 - 2 anos}
      {Analista de Automação e IoT}{Sorocaba, SP}
      \resumeItemListStart
        \resumeItem{Desenvolvi e implementei soluções IoT industriais utilizando ESP32, Raspberry Pi, MQTT e OPC-UA, possibilitando a conectividade de dispositivos em ambientes fabris para otimização e monitoramento de processos produtivos.}
        \resumeItem{Atuei na integração de sistemas SCADA e MES com plataformas cloud, permitindo monitoramento remoto e coleta de dados para análises preditivas.}
        \resumeItem{Colaborei na automação de processos industriais, aplicando C, C++, SCL e Python para controle de sensores, atuadores e equipamentos.}
        \resumeItem{Desenvolvi APIs REST e WebSocket para facilitar a comunicação entre dispositivos de chão de fábrica e aplicações corporativas, garantindo maior eficiência e interoperabilidade.}
        % \resumeItem{}
    \resumeItemListEnd
    
  \resumeSubHeadingListEnd
\vspace{-16pt}

%-----------PROJECTS-----------
% \section{Projetos}
%     \vspace{-5pt}
%     \resumeSubHeadingListStart
%       \resumeProjectHeading
%           {\textbf\faGithub\hspace{4pt}{\href{https://github.com/KauanIzidoro/<repo_name>}{Project Name 1}} $|$ \emph{Tech 1, Tech 2, Tech 3, Tech 4}}{month 2025}
%           \resumeItemListStart
%             \resumeItem{item 1}
%             \resumeItem{item 2}
%             \resumeItem{item 3}
%           \resumeItemListEnd
%           \vspace{-13pt}
    %       {\textbf{Water Safer Monitoring} $|$ \emph{\texttt{C\#}, ASP.NET, SQL Server, Docker, React}}{Outubro 2024}
    %       \resumeItemListStart
    %         \resumeItem{}
    %         \resumeItem{}
    %         \resumeItem{}
    %       \resumeItemListEnd 
    % \resumeSubHeadingListEnd
% \vspace{-15pt}


%
%-----------PROGRAMMING SKILLS-----------
\section{Habilidades Técnicas}
 \begin{itemize}[leftmargin=0.15in, label={}]
    \small{\item{
     \textbf{Linguagens}{: C++, Java, JavaScript, TypeScript, Python, \texttt{C\#}} \\
     \textbf{Frameworks e Ferramentas}{: NestJS, NodeJS, FastAPI, ASP.NET, Spring} \\
     \textbf{Banco de dados}{: MySQL, SQL Server, PostgreSQL, MongoDB, Apache Cassandra} \\
     \textbf{Cloud e DevOps}{: Docker, Kubernetes, Linux, Git, AWS, GCP} \\
     \textbf{Metodologias}{: Agile, Scrum, CI/CD, System Design, Design Patterns} \\
    }} 
 \end{itemize}
 \vspace{-16pt}

%
%-----------CERTIFICATIONS-----------
 \section{Certificações}
    \item Introduction to Data Analytics on Google Cloud \\
    \item Analyze Speech and Language with Google APIs Skill Badge
    \item Build a Secure Google Cloud Network Skill Badge
    \item Build and Deploy Machine Learning Solutions on Vertex AI Skill Badge
    \item Create ML Models with BigQuery ML Skill Badge
    \item Google Cloud Computing Foundations Certificate
    \item Google Cloud Computing Foundations: Data, ML, and AI in Google Cloud
    \item Google Cloud Computing Foundations: Infrastructure in Google Cloud
    \item Google Cloud Computing Foundations: Networking
    \item Introduction to AI and Machine Learning on Google Cloud
    \item Prepare Data for ML APIs on Google Cloud Skill Badge
    \item Set Up an App Dev Environment on Google Cloud Skill Badge
    \item Use Machine Learning APIs on Google Cloud Skill Badge
    \item Implement Load Balancing on Compute Engine Skill Badge
 \vspace{-8pt}

 %-----------Languages-----------
\section{Idiomas}
\begin{itemize}[leftmargin=0.15in, label={}]
% \item Português - Nativo \\
\item Inglês - Proficiência Profissional (B2)
\vspace{-16pt}
\end{itemize}
\end{document}
