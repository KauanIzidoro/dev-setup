\documentclass[letterpaper,11pt]{article}

\usepackage{latexsym}
\usepackage[empty]{fullpage}
\usepackage{titlesec}
\usepackage{marvosym}
\usepackage[usenames,dvipsnames]{color}
\usepackage{verbatim}
\usepackage{enumitem}
\usepackage[hidelinks]{hyperref}
\usepackage{fancyhdr}
\usepackage[english]{babel}
\usepackage{tabularx}
\usepackage{fontawesome5}
\usepackage{multicol}
\setlength{\multicolsep}{-3.0pt}
\setlength{\columnsep}{-1pt}
\input{glyphtounicode}


%----------FONT OPTIONS----------
% sans-serif
% \usepackage[sfdefault]{FiraSans}
% \usepackage[sfdefault]{roboto}
% \usepackage[sfdefault]{noto-sans}
% \usepackage[default]{sourcesanspro}

% serif
% \usepackage{CormorantGaramond}
% \usepackage{charter}


\pagestyle{fancy}
\fancyhf{} % clear all header and footer fields
\fancyfoot{}
\renewcommand{\headrulewidth}{0pt}
\renewcommand{\footrulewidth}{0pt}

% Adjust margins
\addtolength{\oddsidemargin}{-0.6in}
\addtolength{\evensidemargin}{-0.5in}
\addtolength{\textwidth}{1.19in}
\addtolength{\topmargin}{-.7in}
\addtolength{\textheight}{1.4in}

\urlstyle{same}

\raggedbottom
\raggedright
\setlength{\tabcolsep}{0in}

% Sections formatting
\titleformat{\section}{
  \vspace{-4pt}\scshape\raggedright\large\bfseries
}{}{0em}{}[\color{black}\titlerule \vspace{-5pt}]

% Ensure that generate pdf is machine readable/ATS parsable
\pdfgentounicode=1

%-------------------------
% Custom commands
\newcommand{\resumeItem}[1]{
  \item\small{
    {#1 \vspace{-2pt}}
  }
}

\newcommand{\classesList}[4]{
    \item\small{
        {#1 #2 #3 #4 \vspace{-2pt}}
  }
}

\newcommand{\resumeSubheading}[4]{
  \vspace{-2pt}\item
    \begin{tabular*}{1.0\textwidth}[t]{l@{\extracolsep{\fill}}r}
      \textbf{#1} & \textbf{\small #2} \\
      \textit{\small#3} & \textit{\small #4} \\
    \end{tabular*}\vspace{-7pt}
}

\newcommand{\resumeSubSubheading}[2]{
    \item
    \begin{tabular*}{0.97\textwidth}{l@{\extracolsep{\fill}}r}
      \textit{\small#1} & \textit{\small #2} \\
    \end{tabular*}\vspace{-7pt}
}

\newcommand{\resumeProjectHeading}[2]{
    \item
    \begin{tabular*}{1.001\textwidth}{l@{\extracolsep{\fill}}r}
      \small#1 & \textbf{\small #2}\\
    \end{tabular*}\vspace{-7pt}
}

\newcommand{\resumeSubItem}[1]{\resumeItem{#1}\vspace{-4pt}}

\renewcommand\labelitemi{$\vcenter{\hbox{\tiny$\bullet$}}$}
\renewcommand\labelitemii{$\vcenter{\hbox{\tiny$\bullet$}}$}

\newcommand{\resumeSubHeadingListStart}{\begin{itemize}[leftmargin=0.0in, label={}]}
\newcommand{\resumeSubHeadingListEnd}{\end{itemize}}
\newcommand{\resumeItemListStart}{\begin{itemize}}
\newcommand{\resumeItemListEnd}{\end{itemize}\vspace{-5pt}}

%-------------------------------------------
%%%%%%  RESUME STARTS HERE  %%%%%%%%%%%%%%%%%%%%%%%%%%%%


\begin{document}

%----------HEADING----------
% \begin{tabular*}{\textwidth}{l@{\extracolsep{\fill}}r}
%   \textbf{\href{http://sourabhbajaj.com/}{\Large Sourabh Bajaj}} & Email : \href{mailto:sourabh@sourabhbajaj.com}{sourabh@sourabhbajaj.com}\\
%   \href{http://sourabhbajaj.com/}{http://www.sourabhbajaj.com} & Mobile : +1-123-456-7890 \\
% \end{tabular*}

\begin{center}
    {\Huge \scshape Kauan Izidoro} \\ \vspace{1pt}
    São Paulo, Brazil \\ \vspace{1pt}
    \small \raisebox{-0.1\height}\faPhone\ (15) 99618-9346 ~ \href{mailto:x@gmail.com}{\raisebox{-0.2\height}\faEnvelope\  \underline{cnttkauan@gmail.com}} ~ 
    \href{https://linkedin.com/in//}{\raisebox{-0.2\height}\faLinkedin\ \underline{linkedin.com/in/kauanizidoro}}  ~
    \href{https://github.com/}{\raisebox{-0.2\height}\faGithub\ \underline{github.com/KauanIzidoro}}
    \vspace{-8pt}
\end{center}


%-----------EDUCATION-----------
\section{Educação}
  \resumeSubHeadingListStart
    \resumeSubheading
      {Faculdade de Tecnologia SENAI São Paulo}{Jan. 2024 - Dez. 2025}
      {Superior em Análise e Desenvolvimento de Sistemas}{São Paulo, Brazil}
  \resumeSubHeadingListEnd

%------ABOUT-------
\section{Sobre}
    \resumeSubHeadingListStart
        \small{Desenvolvedor Backend com foco na criação e otimização de aplicações que solucionam desafios reais, utilizando tecnologias modernas e boas práticas de engenharia de software. Possuo experiência no desenvolvimento de soluções baseadas em microsserviços, integração de sistemas e implementação de arquiteturas distribuídas, com domínio em tecnologias como Python, FastAPI, TypeScript (NestJS), \texttt{C\#} (ASP.NET), Google Cloud Platform (GCP) e ferramentas como BigQuery, Vertex AI, Pub/Sub e Secret Manager.

        Tenho expertise em bancos de dados relacionais (MySQL, PostgreSQL) e NoSQL (MongoDB, Cassandra), além de experiência em práticas de DevOps, Cloud Computing, Linux, Docker, Kubernetes e integração de sistemas via APIs (REST, GraphQL e WebSocket). Minha atuação é voltada para a entrega de soluções escaláveis, eficientes e alinhadas às necessidades dos negócios.

        Minha formação inclui graduação em Análise e Desenvolvimento de Sistemas pela Faculdade de Tecnologia SENAI e formação técnica em Eletrônica pelo SENAI em parceria com a Robert Bosch VM. Sou apaixonado por tecnologia e busco constantemente aprimorar minhas habilidades para contribuir com projetos inovadores e de alto impacto.}
   \resumeSubHeadingListEnd

%-----------EXPERIENCE-----------
\section{Experiência}
  \resumeSubHeadingListStart

    \resumeSubheading
      {2RP Net Data Driven Company}{Jan. 2024 - Fev. 2025}
      {Desenvolvedor Back-End}{Sorocaba, SP}
      \resumeItemListStart
        \resumeItem{Desenvolvi e implementei microsserviços utilizando Python e FastAPI para facilitar a comunicação entre serviços internos e APIs como BigQuery, Vertex AI, Pub/Sub e Secret Manager na Google Cloud Platform (GCP).}
        \resumeItem{Colaborei na migração de um ambiente em nuvem, criando um ambiente Sandbox para testes e rotinas analíticas, visando agilizar a validação de novas abordagens e a entrega de soluções aos clientes.}
        \resumeItem{Integrei tecnologias de IA, como a Gemini API, para desenvolver um ChatBot com Agente de IA, visando automatizar a documentação e melhorar a acessibilidade às informações de projetos internos.}
        \resumeItem{Utilizei práticas de DevOps, como Docker e Kubernetes, para garantir a escalabilidade e eficiência das aplicações desenvolvidas.}
        \resumeItem{Participei da integração de sistemas utilizando APIs REST e WebSocket, garantindo uma comunicação ágil e eficiente entre serviços.}
      \resumeItemListEnd

    \resumeSubheading
      {Robert Bosch VM}{Jul. 2022 - Dez. 2023}
      {Jovem Aprendiz}{Sorocaba, SP}
      \resumeItemListStart
        \resumeItem{Utilizei módulos SAP ERP (WM, EWM e PM), para executar atividades de movimentação de estoque.}
        \resumeItem{Criei dashboards com ferramentas de Office, como Power BI e Excel, para exibir resultados obtidos a partir de análises da operação logística do setor.}
    \resumeItemListEnd
    
  \resumeSubHeadingListEnd
\vspace{-16pt}

%-----------PROJECTS-----------
\section{Projetos}
    \vspace{-5pt}
    \resumeSubHeadingListStart
      \resumeProjectHeading
          {\textbf{TagSystem} $|$ \emph{\texttt{C\#}, ASP.NET, MySQL, React}}{Dezembro 2024}
          \resumeItemListStart
            \resumeItem{}
            \resumeItem{}
            \resumeItem{}
            \resumeItem{}
          \resumeItemListEnd
          \vspace{-13pt}
      \resumeProjectHeading
          {\textbf{Water Safer Monitoring} $|$ \emph{\texttt{C\#}, ASP.NET, MySQL, React}}{Novembro 2024}
          \resumeItemListStart
            \resumeItem{}
            \resumeItem{}
            \resumeItem{}
          \resumeItemListEnd 
          \vspace{-13pt}
          \resumeProjectHeading
          {\textbf{Transaction Management GUI} $|$ \emph{Java, Eclipse, JavaFX}}{October 2020}
          \resumeItemListStart
            \resumeItem{}
            \resumeItem{}
            \resumeItem{}
          \resumeItemListEnd 
    \resumeSubHeadingListEnd
\vspace{-15pt}


%
%-----------PROGRAMMING SKILLS-----------
\section{Programming Skills}
 \begin{itemize}[leftmargin=0.15in, label={}]
    \small{\item{
     \textbf{Languages}{: Python, Java, C, HTML/CSS, JavaScript, SQL} \\
     \textbf{Developer Tools}{: VS Code, Eclipse, Google Cloud Platform, Android Studio} \\
     \textbf{Technologies/Frameworks}{: Linux, Jenkins, GitHub, JUnit, WordPress} \\
    }}
 \end{itemize}
 \vspace{-16pt}
\end{document}
